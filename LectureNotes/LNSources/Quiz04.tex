\documentclass{beamer}
%
% Choose how your presentation looks.
%
% For more themes, color themes and font themes, see:
% http://deic.uab.es/~iblanes/beamer_gallery/index_by_theme.html
%
\mode<presentation>
{
  \usetheme{Madrid}      % or try Darmstadt, Madrid, Warsaw, ...
  \usecolortheme{seahorse} % or try albatross, beaver, crane, ...
  \usefonttheme{serif}  % or try serif, structurebold, ...
  \setbeamertemplate{navigation symbols}{}
  \setbeamertemplate{caption}[numbered]
} 


\usepackage[english]{babel}
\usepackage{kotex}
%\usepackage[utf8x]{inputenc}

\title[게임수학 - 수시고사5]{ 게임 수학 수시고사 5}
\author{강영민}
\institute{동명대학교}
\date{2015년 2학기}

\begin{document}

% Uncomment these lines for an automatically generated outline.
%\begin{frame}{Outline}
%  \tableofcontents
%\end{frame}


%%%%%%%%%%%%%%%%%%%%%%%%%%%%%%%%%%%%%%%%%%%%%%%%%%%%%%%%%


%%%%%%%%%%%%%%%%%%%%%%%%%%%%%%%%%%%%%%%%%%%%%%%%%%%%%%%%%
\begin{frame}{\small 수시고사 5 - 2015년 11월 10일 (화) $~~$ 학번:$~~~~~~~~~~~~~~~~~~$                이름:  }

\begin{tabular}{|p{11cm}|} \hline
\tiny 2차원 평면 위의 점 $(p_x , p_y )$를 원점 기준으로 $theta$만큼 회전시킨 좌표를 구하는 방법을 다음과 같은 항등식을 활용하여 유도하라.
$$\cos (a+b) = \cos a \cos b - \sin a \sin b, ~~~~~\sin (a+b) = \sin a \cos b + \cos a \sin b$$
\\ \hline \hline
 \\ [30ex] \hline 
\end{tabular}

\end{frame}


%%%%%%%%%%%%%%%%%%%%%%%%%%%%%%%%%%%%%%%%%%%%%%%%%%%%%%%%%

\end{document}


