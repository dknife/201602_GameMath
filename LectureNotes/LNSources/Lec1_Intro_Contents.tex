%%%%%%%%%%%%%%%%%%%%%%%%%%%%%%%%%%%%%%%%%%%%%%%%%%%%%%%%%
\begin{frame}{강의개요}
\begin{itemize}
  \item 교재명: {\small \it 게임프로그래머를 위한} 수학과 OpenGL 프로그래밍
  \item 저자 - 강영민
  \item 출판사 -  도서출판 GS인터비전.
  \item 강의방식
    \begin{itemize}
      \item 강의식 진행
      \item 수시고사 10회
      \item 프로그래밍 구현 실습
    \end{itemize} 
\end{itemize}

%%%%%%%%%%%%%%%%%%%%%%%%%%%%%%%%%%%%%%%%%%%%%%%%%%%%%%%%%
\begin{block}{강의 목표}
게임과 같은 3차원 콘텐츠를 설계하고 제작하는 데에 필요한 기본적인 수학 지식을 습득한다.
\end{block}

\end{frame}


%%%%%%%%%%%%%%%%%%%%%%%%%%%%%%%%%%%%%%%%%%%%%%%%%%%%%%%%%
\begin{frame}{무엇을 다루나}

% Commands to include a figure:
%\begin{figure}
%\includegraphics[width=\textwidth]{your-figure's-file-name}
%\caption{\label{fig:your-figure}Caption goes here.}
%\end{figure}

\begin{table}
\centering
\begin{tabular}{l|c}  \hline
주제 & 내용 \\\hline  \hline
벡터 & 벡터의 개념과 연산 \\ \hline
행렬 & 행렬의 기학적 개념 이해와 응용 \\ \hline
변환 & 변환의 개념 및 행렬 표현 이해 \\ \hline
사원수 & 사원수를 이용한 변환 개념 이해 \\ \hline
카메라 & 카메라 투영 행렬의 이해 \\ \hline
조명과 재질 & 조명 계산 모델의 이해 \\ \hline
충돌 & 기하 객체의 충돌을 감지하는 방법 이해 \\ \hline
\end{tabular}
%\caption{\label{tab:widgets}게임수학에서 다룰 내용}
\end{table}

\end{frame}


%%%%%%%%%%%%%%%%%%%%%%%%%%%%%%%%%%%%%%%%%%%%%%%%%%%%%%%%%
\begin{frame}{마음의 준비}

\begin{itemize}
\item 수식에 겁 먹지 말자
	\begin{itemize}
		\item $ \frac{1}{n} \sum_{i=1}^{n} x_i $은 그냥 $ \frac{x_1 + x_2 + x_3 + x_4 + \cdots + x_n}{n}$이다. 
	\end{itemize}
\item 거의 매주 시험을 친다. 시험 문제는 언제나 미리 알려 준다. 
	\begin{itemize}
		\item 결석은 출석 점수를 못 받아서 무서운 것이 아니라, 시험 문제를 알지 못 하게 되어 무서운 것이다.
	\end{itemize}
\item 출석인정에 대하여
	\begin{itemize}
		\item 사정이 있으면 결석을 하라. 세상에는 수업보다 중요한 일이 많다.
		\item 선택의 책임은 본인 것이다. 출석 인정은 없다. 세상은 그런 곳이다.
		\item 세상에 안 되는 일은 없다. 필요한 경우 진지하게 상담요청을 하라.
	\end{itemize}
\end{itemize}

\end{frame}
