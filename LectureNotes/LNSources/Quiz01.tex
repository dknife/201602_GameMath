\documentclass{beamer}
%
% Choose how your presentation looks.
%
% For more themes, color themes and font themes, see:
% http://deic.uab.es/~iblanes/beamer_gallery/index_by_theme.html
%
\mode<presentation>
{
  \usetheme{Madrid}      % or try Darmstadt, Madrid, Warsaw, ...
  \usecolortheme{seahorse} % or try albatross, beaver, crane, ...
  \usefonttheme{serif}  % or try serif, structurebold, ...
  \setbeamertemplate{navigation symbols}{}
  \setbeamertemplate{caption}[numbered]
} 


\usepackage[english]{babel}
\usepackage{kotex}
%\usepackage[utf8x]{inputenc}

\title[게임수학 - 수시고사1]{ 게임 수학 수시고사 1}
\author{강영민}
\institute{동명대학교}
\date{2015년 2학기}

\begin{document}

% Uncomment these lines for an automatically generated outline.
%\begin{frame}{Outline}
%  \tableofcontents
%\end{frame}


%%%%%%%%%%%%%%%%%%%%%%%%%%%%%%%%%%%%%%%%%%%%%%%%%%%%%%%%%


%%%%%%%%%%%%%%%%%%%%%%%%%%%%%%%%%%%%%%%%%%%%%%%%%%%%%%%%%
\begin{frame}{\small 수시고사 1 - 2015년 9월 15일 (화) $~~$ 학번:$~~~~~~~~~~~~~~~~~~$                이름:  }

\begin{tabular}{|p{5.5cm}|p{5.5cm}|} \hline
\tiny 3차원 벡터 $(v_x,v_y,v_z)$의 길이가 $\sqrt{{v_x}^2 + {v_y}^2 + z^2 }$임을 증명하라. 
&
\tiny 어떤 벡터 (3,2,1)을 정규화하라.\\ \hline \hline

&
 \\ [35ex] \hline 
\end{tabular}

\end{frame}


%%%%%%%%%%%%%%%%%%%%%%%%%%%%%%%%%%%%%%%%%%%%%%%%%%%%%%%%%

\end{document}


